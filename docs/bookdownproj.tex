% Options for packages loaded elsewhere
\PassOptionsToPackage{unicode}{hyperref}
\PassOptionsToPackage{hyphens}{url}
%
\documentclass[
]{book}
\usepackage{amsmath,amssymb}
\usepackage{lmodern}
\usepackage{iftex}
\ifPDFTeX
  \usepackage[T1]{fontenc}
  \usepackage[utf8]{inputenc}
  \usepackage{textcomp} % provide euro and other symbols
\else % if luatex or xetex
  \usepackage{unicode-math}
  \defaultfontfeatures{Scale=MatchLowercase}
  \defaultfontfeatures[\rmfamily]{Ligatures=TeX,Scale=1}
\fi
% Use upquote if available, for straight quotes in verbatim environments
\IfFileExists{upquote.sty}{\usepackage{upquote}}{}
\IfFileExists{microtype.sty}{% use microtype if available
  \usepackage[]{microtype}
  \UseMicrotypeSet[protrusion]{basicmath} % disable protrusion for tt fonts
}{}
\makeatletter
\@ifundefined{KOMAClassName}{% if non-KOMA class
  \IfFileExists{parskip.sty}{%
    \usepackage{parskip}
  }{% else
    \setlength{\parindent}{0pt}
    \setlength{\parskip}{6pt plus 2pt minus 1pt}}
}{% if KOMA class
  \KOMAoptions{parskip=half}}
\makeatother
\usepackage{xcolor}
\IfFileExists{xurl.sty}{\usepackage{xurl}}{} % add URL line breaks if available
\IfFileExists{bookmark.sty}{\usepackage{bookmark}}{\usepackage{hyperref}}
\hypersetup{
  pdftitle={Trees},
  pdfauthor={CS 225 Course Staff},
  hidelinks,
  pdfcreator={LaTeX via pandoc}}
\urlstyle{same} % disable monospaced font for URLs
\usepackage{color}
\usepackage{fancyvrb}
\newcommand{\VerbBar}{|}
\newcommand{\VERB}{\Verb[commandchars=\\\{\}]}
\DefineVerbatimEnvironment{Highlighting}{Verbatim}{commandchars=\\\{\}}
% Add ',fontsize=\small' for more characters per line
\usepackage{framed}
\definecolor{shadecolor}{RGB}{248,248,248}
\newenvironment{Shaded}{\begin{snugshade}}{\end{snugshade}}
\newcommand{\AlertTok}[1]{\textcolor[rgb]{0.94,0.16,0.16}{#1}}
\newcommand{\AnnotationTok}[1]{\textcolor[rgb]{0.56,0.35,0.01}{\textbf{\textit{#1}}}}
\newcommand{\AttributeTok}[1]{\textcolor[rgb]{0.77,0.63,0.00}{#1}}
\newcommand{\BaseNTok}[1]{\textcolor[rgb]{0.00,0.00,0.81}{#1}}
\newcommand{\BuiltInTok}[1]{#1}
\newcommand{\CharTok}[1]{\textcolor[rgb]{0.31,0.60,0.02}{#1}}
\newcommand{\CommentTok}[1]{\textcolor[rgb]{0.56,0.35,0.01}{\textit{#1}}}
\newcommand{\CommentVarTok}[1]{\textcolor[rgb]{0.56,0.35,0.01}{\textbf{\textit{#1}}}}
\newcommand{\ConstantTok}[1]{\textcolor[rgb]{0.00,0.00,0.00}{#1}}
\newcommand{\ControlFlowTok}[1]{\textcolor[rgb]{0.13,0.29,0.53}{\textbf{#1}}}
\newcommand{\DataTypeTok}[1]{\textcolor[rgb]{0.13,0.29,0.53}{#1}}
\newcommand{\DecValTok}[1]{\textcolor[rgb]{0.00,0.00,0.81}{#1}}
\newcommand{\DocumentationTok}[1]{\textcolor[rgb]{0.56,0.35,0.01}{\textbf{\textit{#1}}}}
\newcommand{\ErrorTok}[1]{\textcolor[rgb]{0.64,0.00,0.00}{\textbf{#1}}}
\newcommand{\ExtensionTok}[1]{#1}
\newcommand{\FloatTok}[1]{\textcolor[rgb]{0.00,0.00,0.81}{#1}}
\newcommand{\FunctionTok}[1]{\textcolor[rgb]{0.00,0.00,0.00}{#1}}
\newcommand{\ImportTok}[1]{#1}
\newcommand{\InformationTok}[1]{\textcolor[rgb]{0.56,0.35,0.01}{\textbf{\textit{#1}}}}
\newcommand{\KeywordTok}[1]{\textcolor[rgb]{0.13,0.29,0.53}{\textbf{#1}}}
\newcommand{\NormalTok}[1]{#1}
\newcommand{\OperatorTok}[1]{\textcolor[rgb]{0.81,0.36,0.00}{\textbf{#1}}}
\newcommand{\OtherTok}[1]{\textcolor[rgb]{0.56,0.35,0.01}{#1}}
\newcommand{\PreprocessorTok}[1]{\textcolor[rgb]{0.56,0.35,0.01}{\textit{#1}}}
\newcommand{\RegionMarkerTok}[1]{#1}
\newcommand{\SpecialCharTok}[1]{\textcolor[rgb]{0.00,0.00,0.00}{#1}}
\newcommand{\SpecialStringTok}[1]{\textcolor[rgb]{0.31,0.60,0.02}{#1}}
\newcommand{\StringTok}[1]{\textcolor[rgb]{0.31,0.60,0.02}{#1}}
\newcommand{\VariableTok}[1]{\textcolor[rgb]{0.00,0.00,0.00}{#1}}
\newcommand{\VerbatimStringTok}[1]{\textcolor[rgb]{0.31,0.60,0.02}{#1}}
\newcommand{\WarningTok}[1]{\textcolor[rgb]{0.56,0.35,0.01}{\textbf{\textit{#1}}}}
\usepackage{longtable,booktabs,array}
\usepackage{calc} % for calculating minipage widths
% Correct order of tables after \paragraph or \subparagraph
\usepackage{etoolbox}
\makeatletter
\patchcmd\longtable{\par}{\if@noskipsec\mbox{}\fi\par}{}{}
\makeatother
% Allow footnotes in longtable head/foot
\IfFileExists{footnotehyper.sty}{\usepackage{footnotehyper}}{\usepackage{footnote}}
\makesavenoteenv{longtable}
\usepackage{graphicx}
\makeatletter
\def\maxwidth{\ifdim\Gin@nat@width>\linewidth\linewidth\else\Gin@nat@width\fi}
\def\maxheight{\ifdim\Gin@nat@height>\textheight\textheight\else\Gin@nat@height\fi}
\makeatother
% Scale images if necessary, so that they will not overflow the page
% margins by default, and it is still possible to overwrite the defaults
% using explicit options in \includegraphics[width, height, ...]{}
\setkeys{Gin}{width=\maxwidth,height=\maxheight,keepaspectratio}
% Set default figure placement to htbp
\makeatletter
\def\fps@figure{htbp}
\makeatother
\setlength{\emergencystretch}{3em} % prevent overfull lines
\providecommand{\tightlist}{%
  \setlength{\itemsep}{0pt}\setlength{\parskip}{0pt}}
\setcounter{secnumdepth}{5}
\ifLuaTeX
  \usepackage{selnolig}  % disable illegal ligatures
\fi
\usepackage[]{natbib}
\bibliographystyle{plainnat}

\title{Trees}
\author{CS 225 Course Staff}
\date{2022-10-25}

\begin{document}
\maketitle

{
\setcounter{tocdepth}{2}
\tableofcontents
}
\hypertarget{data-structures-in-c}{%
\chapter{Data Structures in C++}\label{data-structures-in-c}}

\hypertarget{the-pool-of-tears}{%
\chapter{The pool of tears}\label{the-pool-of-tears}}

\begin{Shaded}
\begin{Highlighting}[]
\BuiltInTok{std::}\NormalTok{cout}\OperatorTok{\textless{}\textless{}} \StringTok{"hello harsh"} \OperatorTok{\textless{}\textless{}}\BuiltInTok{std::}\NormalTok{endl}\OperatorTok{;}
\end{Highlighting}
\end{Shaded}

\begin{figure}
\centering
\includegraphics{images/logo.png}
\caption{alt text or image title}
\end{figure}

\hypertarget{a-caucus-race-and-a-long-tale}{%
\chapter{A caucus-race and a long tale}\label{a-caucus-race-and-a-long-tale}}

\hypertarget{trees}{%
\chapter{Trees}\label{trees}}

You're either a botanist or computer scientist if you can talk about trees beyond the \#savethetrees kind of narrative. Unlike trees in real life, Trees in computer science have a root node at the ``top''. So far we've talked about \textbf{linear} data structures. Linear data structures are ordered, meaning that there is a sequence with which our data is ordered and traversed. Trees are a \textbf{non-linear} data structure, meaning that the data isn't organized in a sequential manner. This means that you can visit all the elements on a tree in many different ways based on what type of problem you are trying to solve. For problems where we want to optimize for a certain outcome and don't care about order as much, trees are perfect.

If we take a very high level look at what trees and lists are composed of, it's just nodes and pointers, however, the difference with trees that they are \textbf{hierarchical} meaning that there is a top down organization.
Trees are a hierarchical data structure with a certain set of properties that distinguish it from graphs. Trees are rooted, which means that there is a pointer to the root node and each child node can be reached via the root.

\hypertarget{basic-tree-terminology}{%
\section{Basic tree terminology}\label{basic-tree-terminology}}

(adapted from CS 173)
* Vertex: ``nodes''

\begin{itemize}
\item
  Path: sequence of edges
\item
  Parents: Node \textbf{b, d, x} have Node \textbf{a} as their parent
\item
  Children: \textbf{b, d, x,} are the children of \textbf{a}
\item
  Siblings: \textbf{b, d, x,} are siblings of each other
\item
  Ancestors: \textbf{u} has ancestors \textbf{l, d, a}
\item
  Descendants: \textbf{x} has \textbf{s, m} as its descendants
\item
  Leaves: Vertices with no children
\end{itemize}

\hypertarget{tree-property-height}{%
\section{Tree Property: Height}\label{tree-property-height}}

\begin{itemize}
\tightlist
\item
  \textbf{Computation of the tree height}

  \begin{itemize}
  \tightlist
  \item
    The length of the longest path from the root to the leaf (count edges).
  \item
    If we want to compute recursively:
  \end{itemize}

  height(T) = 1 + max(height(TL), height(TR)), where if height(null) = -1, which might be counter-intuitive but it follows the mathematical definition of tree height
\end{itemize}

\hypertarget{tree-property-binary}{%
\section{Tree Property: Binary}\label{tree-property-binary}}

\begin{itemize}
\tightlist
\item
  A binary tree is either

  \begin{itemize}
  \tightlist
  \item
    T = \{TL, TR, r\}, where TL, TR are binary trees
  \item
    T = \{\} = ~\(\emptyset\)
  \end{itemize}
\end{itemize}

\hypertarget{tree-property-full}{%
\section{Tree Property: Full}\label{tree-property-full}}

\begin{itemize}
\tightlist
\item
  A binary tree is full \emph{if and only if}

  \begin{itemize}
  \tightlist
  \item
    Either: F = \{\}
  \item
    Or: F = \{TL, TR, r\} where TL, TR both have either 0 or 2 children
  \end{itemize}
\item
  \textbf{Theorem}: A binary tree with n data items has n+1 null pointers.
\end{itemize}

\hypertarget{tree-property-perfect}{%
\section{Tree Property: Perfect}\label{tree-property-perfect}}

\begin{itemize}
\tightlist
\item
  A perfect tree Ph is defined by its height

  \begin{itemize}
  \tightlist
  \item
    Ph is a tree of height \textbf{h}, with

    \begin{itemize}
    \tightlist
    \item
      P-1 = \{\}
    \item
      Ph = \{r, Ph-1, Ph-1\} when h\textgreater=0
    \end{itemize}
  \end{itemize}
\end{itemize}

\hypertarget{tree-property-complete}{%
\section{Tree Property: Complete}\label{tree-property-complete}}

(as defined in data structures)
* A complete tree is
* A perfect tree except for the last level

\begin{itemize}
\item
  All leaves must be pushed to the \textbf{left}
\item
  Or, recursively, a complete tree \textbf{Ch} of height \textbf{h} is

  \begin{itemize}
  \item
    C-1 = \{\}
  \item
    Ch = \{r, TL, TR\} where

    \begin{itemize}
    \tightlist
    \item
      Either: TL = Ch-1 and TR = Ph-2
      Or:TL = Ph-1 and TR = Ch-1
    \end{itemize}
  \end{itemize}
\end{itemize}

\begin{itemize}
\tightlist
\item
  Full does not imply perfect, so as complete does not imply perfect
\item
  Not full implies not perfect, thus perfect implies full; perfect also implies complete too.
\end{itemize}

\hypertarget{tree-traversals}{%
\section{Tree Traversals}\label{tree-traversals}}

(practice them here: \url{https://yongdanielliang.github.io/animation/web/BST.html})
* Pre-Order: process the data first, then left child, then the right child
* In-Order: left child, process the data, right child
* Post-Order: left child, right child, process the data last

\begin{Shaded}
\begin{Highlighting}[]
\DataTypeTok{void}\NormalTok{ BinaryTree}\OperatorTok{\textless{}}\NormalTok{T}\OperatorTok{\textgreater{}::}\NormalTok{preOrder}\OperatorTok{(}\NormalTok{TreeNode }\OperatorTok{*}\NormalTok{ cur}\OperatorTok{)} \OperatorTok{\{}
    \ControlFlowTok{if} \OperatorTok{(}\NormalTok{cur }\OperatorTok{!=}\NormalTok{ NULL}\OperatorTok{)} \OperatorTok{\{}
\NormalTok{        func}\OperatorTok{(}\NormalTok{curr}\OperatorTok{{-}\textgreater{}}\NormalTok{data}\OperatorTok{);}
\NormalTok{        preOrder}\OperatorTok{(}\NormalTok{curr}\OperatorTok{{-}\textgreater{}}\NormalTok{left}\OperatorTok{);}
\NormalTok{        preOrder}\OperatorTok{(}\NormalTok{curr}\OperatorTok{{-}\textgreater{}}\NormalTok{right}\OperatorTok{);}
    \OperatorTok{\}}
\OperatorTok{\}}

\DataTypeTok{void}\NormalTok{ BinaryTree}\OperatorTok{\textless{}}\NormalTok{T}\OperatorTok{\textgreater{}::}\NormalTok{inOrder}\OperatorTok{(}\NormalTok{TreeNode }\OperatorTok{*}\NormalTok{ cur}\OperatorTok{)} \OperatorTok{\{}
    \ControlFlowTok{if} \OperatorTok{(}\NormalTok{cur }\OperatorTok{!=}\NormalTok{ NULL}\OperatorTok{)} \OperatorTok{\{}
\NormalTok{        preOrder}\OperatorTok{(}\NormalTok{curr}\OperatorTok{{-}\textgreater{}}\NormalTok{left}\OperatorTok{);}
\NormalTok{        func}\OperatorTok{(}\NormalTok{curr}\OperatorTok{{-}\textgreater{}}\NormalTok{data}\OperatorTok{);}
\NormalTok{        preOrder}\OperatorTok{(}\NormalTok{curr}\OperatorTok{{-}\textgreater{}}\NormalTok{right}\OperatorTok{);}
    \OperatorTok{\}}
\OperatorTok{\}}

\DataTypeTok{void}\NormalTok{ BinaryTree}\OperatorTok{\textless{}}\NormalTok{T}\OperatorTok{\textgreater{}::}\NormalTok{inOrder}\OperatorTok{(}\NormalTok{TreeNode }\OperatorTok{*}\NormalTok{ cur}\OperatorTok{)} \OperatorTok{\{}
    \ControlFlowTok{if} \OperatorTok{(}\NormalTok{cur }\OperatorTok{!=}\NormalTok{ NULL}\OperatorTok{)} \OperatorTok{\{}
\NormalTok{        preOrder}\OperatorTok{(}\NormalTok{curr}\OperatorTok{{-}\textgreater{}}\NormalTok{left}\OperatorTok{);}
\NormalTok{        preOrder}\OperatorTok{(}\NormalTok{curr}\OperatorTok{{-}\textgreater{}}\NormalTok{right}\OperatorTok{);}
\NormalTok{        func}\OperatorTok{(}\NormalTok{curr}\OperatorTok{{-}\textgreater{}}\NormalTok{data}\OperatorTok{);}
    \OperatorTok{\}}
\OperatorTok{\}}
\end{Highlighting}
\end{Shaded}

\hypertarget{searching-trees}{%
\section{Searching Trees}\label{searching-trees}}

\begin{itemize}
\item
  BFS: breadth first search: visits nodes at each level (level-order traversal): use a queue
\item
  DFS: depth first search: find the endpoint of the path quickly (in order, pre order or post order): use a stack
\item
  Traversal vs Search: traverse visits every node vs search visits nodes until you find what you want
\end{itemize}

\hypertarget{delete-and-insert}{%
\section{Delete and Insert}\label{delete-and-insert}}

\end{document}
